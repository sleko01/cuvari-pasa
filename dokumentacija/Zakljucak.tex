\chapter{Zaključak i budući rad}
		
		%\textbf{\textit{dio 2. revizije}}\\
		
		 %\textit{U ovom poglavlju potrebno je napisati osvrt na vrijeme izrade projektnog zadatka, koji su tehnički izazovi prepoznati, jesu li riješeni ili kako bi mogli biti riješeni, koja su znanja stečena pri izradi projekta, koja bi znanja bila posebno potrebna za brže i kvalitetnije ostvarenje projekta i koje bi bile perspektive za nastavak rada u projektnoj grupi.}
		
		 %\textit{Potrebno je točno popisati funkcionalnosti koje nisu implementirane u ostvarenoj aplikaciji.}
		
		Zadatak našeg tima bio je razvoj web aplikacije za traženje i pružanje usluga čuvanja pasa te međusobnu interakciju korisnika prije početka i nakon završetka čuvanja. U protekla 3 mjeseca, svakog tjedna smo radili na razvoju projekta te time ostvarili glavninu funkcionalnosti koje su bile na početku planirane, a usput smo stekli brojna nova i korisna iskustva. Sami razvoj projekta bio je podijeljen u dvije faze, od kojih je prva bila više organizacijske naravi, dok je druga bila potpuno fokusirana na razvoj i implementaciju samog rješenja.\\
		Na početku prve faze slijedilo je okupljanje tima te međusobno upoznavanje, a zatim smo dobili uvid u projektni zadatak koji je bilo potrebno provesti u djelo. Potom smo uz nekoliko timskih sastanaka putem platforme Discord razjasnili i prokomentirali samu suštinu problema te smo na vrlo visokoj razini apstrakcije izradili izgled nekih stranica aplikacije kako bi svi članovi tima dobili dojam o tome kako bi naša aplikacija otprilike funkcionirala. Nakon toga smo krenuli na podjelu poslova i organizaciju u pojedinim podtimovima. Tim je bio podijeljen na podtimove zadužene za razvoj frontenda i backenda te podtim koji se bavi povezivanjem odgovarajućih frontend i backend komponenti. Detaljnije funkcionalnosti koje aplikacija mora obavljati razradili smo kroz funkcionalne zahtjeve i obrasce uporabe te pripadajuće dijagrame koji su nam uvelike pomogli i olakšali daljnji razvoj aplikacije. Pošto većina članova tima prethodno nije bila upoznata s pojedinim tehnologijama koje smo planirali koristiti za razvoj, kroz niz tehničkih radionica i samostalno istraživanje polako smo se upoznavali sa potrebnim alatima i načinom na koji ih trebamo koristiti. Na kraju prve faze ostvarili smo generičke funkcionalnosti kao što su prijava i registracija u sustav te pripadajući dizajn i izgled tih stranica.\\
		Dolazimo do druge faze u kojoj stvari postaju malo zahtjevnije zbog nekih faktora kao što su količina funkcionalnosti koje je još bilo potrebno ostvariti te vremenski period koji nam je bio na raspolaganju za njihovo ostvarenje. U narednim tjednima rad na aplikaciji i samoj dokumentaciji bio je mnogo intenzivniji. Preostalo je dokumentirati podosta poglavlja te izraditi dijagrame koji odgovaraju radu naše aplikacije. Uz to, kao i uvijek, naišli smo i na neke probleme u razvoju koji su iziskivali puno vremena da se nađe odgovarajuće rješenje. Usprkos tome, na vrijeme smo bili gotovi s ostvarenjem planiranih funkcionalnosti naše aplikacije i nakon toga preostalo je dovršiti izgled same aplikacije te neke sitne preinake kako bi sve bilo ispravno i u funkciji.\\
		Na samom kraju, zadovoljni smo onime što smo postigli i napravili. Svakako najbitnije od svega je iskustvo koje smo stekli tokom rada na ovom projektu. Timski rad te međusobna komunikacija i organizacija je ono što nam je pomoglo da dođemo do završnog proizvoda. Izrada samog projekta vjerojatno bi bila dosta brža i bezbolnija da smo prethodno bili upoznati s korištenim tehnologijama. No, zato smo kroz ovaj projekt dobili potrebna znanja i vjetar u leđa za buduće projekte.\\
		
		
		\eject 