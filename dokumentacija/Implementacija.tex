\chapter{Implementacija i korisničko sučelje}
		
		
		\section{Korištene tehnologije i alati}
			
			%\textbf{\textit{dio 2. revizije}}
			
			 %\textit{Detaljno navesti sve tehnologije i alate koji su primijenjeni pri izradi dokumentacije i aplikacije. Ukratko ih opisati, te navesti njihovo značenje i mjesto primjene. Za svaki navedeni alat i tehnologiju je potrebno \textbf{navesti internet poveznicu} gdje se mogu preuzeti ili više saznati o njima}.
			
			Frontend aplikacije pisan je u programskom jeziku \textbf{\href{https://www.javascript.com/}{JavaScript}} uz pomoć biblioteke \textbf{\href{https://reactjs.org/}{React}}. React je  open-source JavaScript biblioteka za izgradnju komponenti korisničkog sučelja te je odgovorna samo za prezentacijski sloj aplikacije. Glavni cilj Reacta je razvoj korisničkog sučelja koje poboljšava brzinu aplikacija, pa stoga koristi virtualni DOM (Document Object Model) koji je brži od običnog DOM-a. Aplikacije pisane u Reactu su jednostranične što također pridonosi brzini. Korištenjem komponenti u Reactu, poboljšava se čitljivost i olakšava se održavanje većih aplikacija. React je razvijen od strane Facebooka te korišten i u njihovim aplikacijama poput Instagrama i WhatsAppa.\\
			Kao razvojno okruženje za frontend aplikacije korišten je \textbf{\href{https://code.visualstudio.com/}{Visual Studio Code}}. Visual Studio Code je lagan, ali moćan uređivač izvornog koda razvijen od strane tvrtke Microsoft. Neke od glavnih značajki alata su podrška za ispravljanje pogrešaka, isticanje sintakse, inteligentno dovršavanje koda i mnoge druge. Dolazi s ugrađenom podrškom za JavaScript, TypeScript i Node.js, a ima i bogat ekosustav proširenja za druge jezike i okruženja kao što su C++, C\#, Java, Python, PHP, Go, .NET itd.\\
			Backend aplikacije pisan je u programskom jeziku \textbf{\href{https://www.java.com/en/}{Java}} uz pomoć radnog okvira \textbf{\href{https://spring.io/projects/spring-boot}{Spring Boot}}. Spring Boot je vrlo popularan radni okvir za izgradnju samostojećih aplikacija koje koriste Spring. Spring Boot je specijalizacija radnog okvira Spring razvijen s ciljem jednostavnijeg i bržeg oblikovanja web aplikacija, pa stoga u svojoj automatskoj konfiguraciji olakšava posao programeru jer neke stvari, kao npr. servleti, rad s JSON datotekama, rad s bazama podataka itd., koje su karakteristične za većinu web aplikacja ima automatski podešeno. \\
			Kao razvojno okruženje za backend aplikacije korišten je \textbf{\href{https://www.jetbrains.com/idea/}{IntelliJ IDEA}}. IntelliJ IDEA je integrirano razvojno okruženje (IDE) razvijeno od strane tvrtke JetBrains. Samo razvojno okruženje pisano je u Javi s ciljem poboljšanog razvoja softvera u jezicima Java, Kotlin, Groovy i sl. Ovaj IDE pruža značajke kao što su dovršavanje koda analizom konteksta, navigacija u kodu pri čemu je moguće izravno skakanje na klasu ili deklaraciju u kodu, refaktoriranje koda, otklanjanje pogrešaka i brojne druge. IntelliJ IDEA također podržava dodatke pomoću kojih se može ostvariti dodatna funkcionalnost. Dodaci se mogu preuzeti i instalirati putem njihovog web repozitorija dodataka ili putem IDE-ove ugrađene opcije instaliranja dodataka.\\
			Baza podataka je izvedena u \textbf{\href{https://www.postgresql.org/}{PostgreSQL-u}}. PostgreSQL je open-source sustav za upravljanje relacijskim bazama podataka (RDBMS) kojim se proširuje funkcionalnost SQL-a. PostgreSQL nudi transakcije s okidačima, stranim ključevima, pohranjenim procedurama, automatski ažuriranim prikazima i sl. Transakcije također imaju svojstva atomarnosti, konzistentnosti, izolacije i izdržljivosti (ACID). PostgreSQL dizajniran je da izdrži različita radna opterećenja, od pojedinačnih računa-la, pa sve do skladišta podataka ili web usluga s mnogo istodobnih korisnika. \\
			Kao okruženje za upravljanje bazom podataka korišten je \textbf{\href{https://www.pgadmin.org/}{pgAdmin}}. pgAdmin je open-source grafički alat za administrativno upravljanje PostgreSQL bazama podataka.\\
			Sama dokumentacija je pisana u jeziku \textbf{\href{https://www.latex-project.org/}{LaTeX}}. LaTeX je jezik za pisanje strukturiranih tekstova profesionalne kvalitete. Za razliku od nekih programa za obradu teksta s grafičkim sučeljem poput Microsoft Worda, dokumenti se u LaTeXu pišu kao običan tekst s dodanom semantičkom strukturom te se time postiže usredotočenost na sadržaj, ujednačenost izgleda te brži i stabilniji rad.\\
			Kao okruženje za pisanje dokumentaciju korišten je \textbf{\href{https://www.texstudio.org/}{TeXstudio}}. TeXstudio je open-source integrirano okruženje za izradu LaTeX dokumenata. Posjeduje brojne znač-ajke kao što su pogled na strukturu dokumenta, napredno isticanje sintakse, interaktivna provjera pravopisa, gramatike i referenci, jasan pogled na upozorenja i greške u dokumentu itd.\\
			Za izradu UML dijagrama unutar dokumentacije korišteno je okruženje \textbf{\href{https://astah.net/products/astah-uml/}{Astah UML}}. Astah UML je grafički alat koji je jednostavan za rukovanje te se koristi za izradu potrebnih UML dijagrama. Ima podršku za stvaranje brojnih dijagrama, a samo neki od njih su dijagrami obrazaca uporabe, sekvencijski dijagrami, dijagrami razreda, dijagrami stanja, dijagrami aktivnosti, dijagrami komponenti te dijagrami razmještaja za čiju je izradu alat i korišten.\\
			Za izradu dijagrama baze podataka unutar dokumentacije korišten je online alat \textbf{\href{https://drawsql.app/}{DrawSQL}}.\\
			Kao sustav za upravljanje verzijama projekta korišten je \textbf{\href{https://git-scm.com/}{Git}}. Git je open-source distribuirani sustav za upravljanje različitim verzijama datoteka. Obično se koristi za koordinaciju rada među programerima koji zajednički razvijaju neki softver. Cilj Gita je brzina, integritet podataka i podrška za distribuirane i nelinearne tijekove rada. Svaki Git direktorij na bilo kojem računalu je spremište s potpunom poviješću i punim mogućnostima praćenja verzija.\\
			Udaljeni repozitorij projekta dostupan je na web platformi u oblaku \textbf{\href{https://gitlab.com/}{GitLab}}. GitLab je open-source end-to-end platforma u oblaku za razvoj softvera s ugrađenom kontrolom verzija, praćenjem problema, pregledom koda, CI/CD-om i više.\\
			Za deploy aplikacije korišten je \textbf{\href{https://render.com/}{Render}}. Render je objedinjeni sustav u oblaku koji služi za izgradnju i pokretanje web aplikacija i aplikacija općenito. Pruža pogodnosti poput besplatnih TLS (Transport Layer Security) certifikata, globalnog CDN-a (Content Delivery Network), DDoS (Distributed Denial of Service) zaštite, privatnih mreža i automatske implementacije iz Gita.\\
			Za komunikaciju u timu korišten je \textbf{\href{https://discord.com/}{Discord}}. Discord je društvena platforma na kojoj korisnici imaju mogućnost komuniciranja glasovnim pozivima, videopozivima, tekstualnim porukama, medijima i datotekama u privatnim razgovorima ili kao dio zajednica koje se nazivaju serveri. Server je skup soba za razgovor i glasovnih kanala te mu je moguće pristupiti putem pozivnice.\\
			
			
			\eject 
		
	
		\section{Ispitivanje programskog rješenja}
			
		
			\subsection{Ispitivanje komponenti}
			Za testiranje backenda naše aplikacije koristili smo JUnit testove. Svi su testovi bili uspješni, što se vidi na slici 5.1. U sljedećim potpoglavljima prikazani su pojedinačni testovi komponenta.
			
			\begin{figure}[H]
				\centering
				\includegraphics[width=12cm]{slike/testoviKomponenti}
				\caption{Uspješni testovi komponenata}
				\label{fig:Testovi-komponenta}
			\end{figure}
			
				\subsubsection{Test 1: Stvaranje korisnika}
			
				U ovom testu napravili poslali smo bazi gotovi objekt koji koristi za zapisivanje novog korisnika. Za provjeru testa dohvatili smo id iz baze i provjerili je li se taj korisnik ispravno zapisao u bazu.
				
				\begin{figure}[H]
					\centering
					\includegraphics[width=12cm]{slike/stvaranjeKorisnika}
					\caption{Test 1: Stvaranje korisnika}
					\label{fig:Test-1}
				\end{figure}
			
				\subsubsection{Test 2: Stvaranje postojećeg korisnika}
				
				U ovom testu poslali smo bazi korisnika s korisničkim imenom koje već postoji u bazi. Potom smo provjerili baca li baza dobru vrstu iznimke te je li tekst te iznimke ispravan.
				
				\begin{figure}[H]
					\centering
					\includegraphics[width=12cm]{slike/stvaranjePostojecegKorisnika}
					\caption{Test 2: Stvaranje postojećeg korisnika}
					\label{fig:Test-2}
				\end{figure}
			
				\subsubsection{Test 3: Registracija novog psa}
				
				U ovom testu poslali smo bazi gotov objekt za registraciju novog psa. Zatim smo dohvatili sve pse korisnika kojem smo dodali psa, pa smo provjerili postoji li u toj listi pasa pas s istim imenom kao onaj kojeg smo poslali.
				
				\begin{figure}[H]
					\centering
					\includegraphics[width=12cm]{slike/registracijaNovogPsa}
					\caption{Test 3: Registracija novog psa}
					\label{fig:Test-3}
				\end{figure}
			
				\subsubsection{Test 4: Stvaranje psa s pogrešnim datumom rođenja}
				
				U ovom testu poslali smo bazi gotov objekt za registraciju novog psa, ali s datumom rođenja u budućnosti. Zatim smo provjerili baca li baza dobru vrstu iznimke te je li tekst te iznimke ispravan.
				
				\begin{figure}[H]
					\centering
					\includegraphics[width=12cm]{slike/stvaranjePsaSKrivimDatumom}
					\caption{Test 4: Stvaranje psa s pogrešnim datumom rođenja}
					\label{fig:Test-4}
				\end{figure}
			
				\subsubsection{Test 5: Davanje uloge administratora korisniku}
				
				U ovom testu poslali smo bazi id korisnika kojemu želimo dati ulogu administratora. Zatim smo korisnika s tim id-om dohvatili te provjerili odgovara li mu identifikacijski broj uloge ulozi administratora.
				
				\begin{figure}[H]
					\centering
					\includegraphics[width=12cm]{slike/davanjeAdminaKorisniku}
					\caption{Test 5: Davanje uloge administratora korisniku}
					\label{fig:Test-5}
				\end{figure}
			
				\subsubsection{Test 6: Blokiranje korisnika}
				
				U ovom testu poslali smo bazi id korisnika kojega želimo blokirati. Zatim smo korisnika s tim id-om dohvatili te provjerili vrijednost atributa blocked.
				
				\begin{figure}[H]
					\centering
					\includegraphics[width=12cm]{slike/blokiranjeKorisnika}
					\caption{Test 6: Blokiranje korisnika}
					\label{fig:Test-6}
				\end{figure}
			
			\subsection{Ispitivanje sustava}
			
			 Ispitivanje sustava proveli smo pomoću Selenium ispita. Koristili smo Selenium WebDriver za preglednik Google Chrome.
			 
			 \subsubsection{Ispitni slučaj 1: Prijavljivanje korisnika}
			 
			 \begin{packed_item}
			 	
			 	\item \textbf{Ulaz: } Korisničko ime i lozinka korisnika 
			 	\item  \textbf{Očekivani izlaz:} Nakon prijave, nalazit ćemo se na početnoj stranici, što je u našem slučaju http://localhost:3000/, s obzirom da smo testove obavljali lokalno
			 	\item  \textbf{Koraci:}
			 	
			 	\item[] \begin{packed_enum}
			 		
			 		\item Korisnik odabere gumb 'Prijava'
			 		\item Korisnik unese korisničko ime
			 		\item Korisnik unese lozinku
			 		\item Korisnik odabere gumb 'Prijavi se'
			 		
			 	\end{packed_enum}
			 \end{packed_item}
			 
			 \begin{figure}[H]
			 	\centering
			 	\includegraphics[width=12cm]{slike/prijavljivanjeKorisnika}
			 	\caption{Ispitni slučaj 1: Prijavljivanje korisnika}
			 	\label{fig:Ispitni-slucaj-1}
			 \end{figure}
		 
		 	\subsubsection{Ispitni slučaj 2: Dodavanje novog administratora}
		 	
		 	\begin{packed_item}
		 		
		 		\item \textbf{Ulaz: } Korisničko ime i lozinka administratora 
		 		\item  \textbf{Očekivani izlaz:} Dobit ćemo obavijest od stranice koja nas obavještava da je korisnik uspješno dodan kao administrator
		 		\item  \textbf{Koraci:}
		 		
		 		\item[] \begin{packed_enum}
		 			
		 			\item Korisnik odabere gumb 'Prijava'
		 			\item Korisnik unese korisničko ime
		 			\item Korisnik unese lozinku
		 			\item Korisnik odabere gumb 'Prijavi se'
		 			\item Korisnik, koji se nalazi na početnoj stranici, pozicionira miš iznad svojeg korisničkog imena kako bi otkrio padajući izbornik
		 			\item Korisnik odabire gumb 'Moderacija'
		 			\item Korisnik odabire gumb 'Dodaj admina'
		 			\item Korisnik provjerava tekst obavijesti
		 			
		 		\end{packed_enum}
		 	\end{packed_item}
		 	
		 	\begin{figure}[H]
		 		\centering
		 		\includegraphics[width=12cm]{slike/dodavanjeAdmina}
		 		\caption{Ispitni slučaj 2: Dodavanje novog administratora}
		 		\label{fig:Ispitni-slucaj-2}
		 	\end{figure}
	 	
	 		\subsubsection{Ispitni slučaj 3: Registracija novog psa}
	 		
	 		\begin{packed_item}
	 			
	 			\item \textbf{Ulaz: } Korisničko ime, lozinka, ime psa, datum rođenja psa, identifikacijski broj vrste psa
	 			\item  \textbf{Očekivani izlaz:} Korisnik će se nakon izvođenja nalaziti na stranici korisnikovih pasa, gdje će ga nakon registracije preusmjeriti samo ukoliko je registracija uspješna
	 			\item  \textbf{Koraci:}
	 			
	 			\item[] \begin{packed_enum}
	 				
	 				\item Korisnik odabere gumb 'Prijava'
	 				\item Korisnik unese korisničko ime
	 				\item Korisnik unese lozinku
	 				\item Korisnik odabere gumb 'Prijavi se'
	 				\item Korisnik, koji se nalazi na početnoj stranici, pozicionira miš iznad svojeg korisničkog imena kako bi otkrio padajući izbornik
	 				\item Korisnik odabire gumb 'Moji psi'
	 				\item Korisnik odabire gumb 'Dodaj psa'
	 				\item Korisnik unosi ime i datum rođenja psa
	 				\item Korisnik odabire vrstu psa
	 				\item Korisnik odabire gumb 'Registracija psa'
	 				
	 			\end{packed_enum}
	 		\end{packed_item}
	 		
	 		\begin{figure}[H]
	 			\centering
	 			\includegraphics[width=12cm]{slike/registracijaPsa}
	 			\caption{Ispitni slučaj 3: Registracija novog psa}
	 			\label{fig:Ispitni-slucaj-3}
	 		\end{figure}
 		
 			\subsubsection{Ispitni slučaj 4: Dodavanje nove ponude}
 			
 			\begin{packed_item}
 				
 				\item \textbf{Ulaz: }
 				
 				\item[] \begin{packed_item}
 					
 					\item Korisničko ime i lozinka
 					\item Datum i vrijeme početka i kraja čuvanja, adresa, željeni broj pasa i željena dob psa
 					 					
 				\end{packed_item}
 			
 				\item  \textbf{Očekivani izlaz:} Korisnik će se nakon izvođenja nalaziti na stranici korisnikovih oglasa, gdje će biti preusmjeren samo ako će unos biti uspješan
 				\item  \textbf{Koraci:}
 				
 				\item[] \begin{packed_enum}
 					
 					\item Korisnik odabere gumb 'Prijava'
 					\item Korisnik unese korisničko ime
 					\item Korisnik unese lozinku
 					\item Korisnik odabere gumb 'Prijavi se'
 					\item Korisnik odabire gumb 'Oglasi'
 					\item Korisnik odabire gumb 'Dodaj novi oglas'
 					\item Korisnik unosi datum i vrijeme početka i kraja čuvanja, adresu, željeni broj pasa i željenu dob psa
 					\item Korisnik odabire gumb 'Predaj'
 					
 				\end{packed_enum}
 			\end{packed_item}
 			
 			\begin{figure}[H]
 				\centering
 				\includegraphics[width=12cm]{slike/dodavanjeNovePonude}
 				\caption{Ispitni slučaj 4: Dodavanje nove ponude}
 				\label{fig:Ispitni-slucaj-4}
 			\end{figure}
 		
 			\subsubsection{Ispitni slučaj 5: Neispravna prijava}
 			
 			\begin{packed_item}
 				
 				\item \textbf{Ulaz: } Neispravno korisničko ime i/ili lozinka
 				\item  \textbf{Očekivani izlaz:} Dobit ćemo obavijest od aplikacije s tekstom 'Unauthorized'
 				\item  \textbf{Koraci:}
 				
 				\item[] \begin{packed_enum}
 					
 					\item Korisnik odabere gumb 'Prijava'
 					\item Korisnik unese korisničko ime
 					\item Korisnik unese lozinku
 					\item Korisnik odabere gumb 'Prijavi se'
 					
 				\end{packed_enum}
 			\end{packed_item}
 			
 			\begin{figure}[H]
 				\centering
 				\includegraphics[width=12cm]{slike/neispravnaPrijava}
 				\caption{Ispitni slučaj 5: Neispravna prijava}
 				\label{fig:Ispitni-slucaj-5}
 			\end{figure}
			
			\eject 
		
		
		\section{Dijagram razmještaja}
			
			%\textbf{\textit{dio 2. revizije}}
			
			 %\textit{Potrebno je umetnuti \textbf{specifikacijski} dijagram razmještaja i opisati ga. Moguće je umjesto specifikacijskog dijagrama razmještaja umetnuti dijagram razmještaja instanci, pod uvjetom da taj dijagram bolje opisuje neki važniji dio sustava.}
			 
			 Dijagram razmještaja identificira fizičke i virtualne čvorove koji su prisutni u topologiji sustava te opisuje programsku potporu u implementaciji sustava. Sam sustav baziran je na odnosu klijent-poslužitelj + HTTP protokol za komunikaciju između njih. Cijela poslužiteljska strana ostvarena je pomoću servisa Render.
			 
			 \begin{figure}[H]
			 	\centering
			 	\includegraphics[width=15cm]{slike/Dijagram razmjestaja}
			 	\caption{Dijagram razmještaja}
			 	\label{fig:Activity-Diagram}
			 \end{figure}
			
			\eject 
		
		\section{Upute za puštanje u pogon}
			
			\subsection{Konfiguracija poslužitelja baze podataka}
			
			Na poslužitelju Render potrebno je konfigurirati PostgreSQL bazu podataka. Potrebno je postaviti ime baze, korisničko ime korisnika baze, regiju postaviti na Frankfurt (EU Central) i kliknuti Create Database. Pri prvom pokretanju backenda, automatski će se kreirati svi elementi baze Flyway migracijama.
			
			\begin{figure}[H]
				\centering
				\includegraphics[width=12cm]{slike/konfiguracijaBazeNaRenderu}
				\caption{Konfiguracija baze na Renderu}
				\label{fig:Baza-Render}
			\end{figure}
			
			\subsection{Konfiguracija backenda}
			
			
			Na poslužitelju Render potrebno je konfigurirati backend. Potrebno je povezati GitLab račun s Renderom. Nakon toga potrebno je kreirati novi servis i odabrati projekt Čuvari pasa. Za regiju je potrebno odabrati Frankfurt (EU Central), za granu main, za root directory primavara-be. Environment je potrebno postaviti na Docker. Dockerfile Path je ./docker/maven/Dockerfile, a Docker Build Context Directory je . Potrebno je postaviti environment varijable sa slike 5.3 i nakon toga kliknuti Create Web Service.
			
			\begin{figure}[H]
				\centering
				\includegraphics[width=12cm]{slike/konfiguracijaBackenda}
				\caption{Konfiguracija backenda na Renderu}
				\label{fig:Backend-Render}
			\end{figure}
			
			\subsection{Konfiguracija frontenda}
			
			
			Na poslužitelju Render potrebno je konfigurirati frontend. Potrebno je povezati GitLab račun s Renderom. Nakon toga potrebno je kreirati novi servis i odabrati projekt Čuvari pasa. Za regiju je potrebno odabrati Frankfurt (EU Central), za granu main, za root directory primavara-fe. Environment je potrebno postaviti na Node. Build command je potrebno postaviti na yarn build, a start command na yarn start-prod. Potrebno je postaviti environment varijable sa slike 5.4 i nakon toga kliknuti Create Web Service.
			
			\begin{figure}[H]
				\centering
				\includegraphics[width=12cm]{slike/konfiguracijaFrontenda}
				\caption{Konfiguracija frontenda na Renderu}
				\label{fig:Frontend-Render}
			\end{figure}
			\eject 